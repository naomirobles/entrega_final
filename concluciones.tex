\section{Conclusiones}

El desarrollo del presente proyecto permitió analizar y abordar una problemática actual en el ámbito académico: la desconexión existente entre las aplicaciones tradicionales de gestión de tareas y los recursos educativos necesarios para llevar a cabo dichas actividades de manera efectiva. A partir de este análisis, se propuso una solución integral que combina la organización de tareas con un motor de recomendaciones educativas, alineando la planeación y la ejecución del trabajo académico en un solo entorno.

La propuesta de una aplicación web responsiva con recomendaciones basadas en análisis de palabras clave demuestra cómo las técnicas de procesamiento de lenguaje natural pueden integrarse de manera práctica para apoyar el aprendizaje autónomo y fomentar el desarrollo de competencias previas como el pensamiento crítico, la alfabetización digital y la gestión del tiempo. De esta forma, el sistema no sustituye el esfuerzo intelectual del usuario, sino que lo acompaña y orienta hacia recursos relevantes y confiables.

Asimismo, la definición detallada de requerimientos funcionales y no funcionales, junto con los diagramas de arquitectura, procesos y casos de uso, permitió establecer una base sólida para el desarrollo del sistema, asegurando aspectos clave como rendimiento, seguridad, usabilidad y mantenibilidad. La planeación por sprints y la estimación de costos proporcionan una visión realista de la viabilidad técnica y económica del proyecto.

En términos de calidad de software, la incorporación de métricas de rendimiento web mediante herramientas como Vercel Speed Insights resalta la importancia de evaluar la experiencia real del usuario como un criterio fundamental de calidad. La medición continua y basada en datos permite detectar regresiones, priorizar mejoras técnicas y garantizar que el sistema cumpla con los niveles de servicio esperados.

Finalmente, este proyecto sienta las bases para futuras extensiones, como la mejora del motor de recomendaciones mediante modelos más avanzados de aprendizaje automático, la incorporación de nuevas fuentes de contenido educativo y el despliegue en entornos productivos a mayor escala. En conjunto, la solución propuesta representa un enfoque innovador y éticamente responsable para apoyar la gestión de tareas y el aprendizaje en contextos académicos y profesionales.
