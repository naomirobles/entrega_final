\section{Pruebas de Calendario}

\subsection{Navegación del Calendario (e2e/calendar/navigation.spec.ts)}
\begin{itemize}
    \item \textbf{Acceso al calendario autenticado:} Verifica que un usuario autenticado pueda acceder a la página del calendario.
    \item \textbf{Navegación entre meses:} Comprueba que el usuario pueda moverse al mes siguiente y anterior utilizando los botones de navegación del calendario.
\end{itemize}

\subsection{Vistas del Calendario (e2e/calendar/views.spec.ts)}
\begin{itemize}
    \item \textbf{Cambio entre vistas:} Asegura que el usuario pueda cambiar entre la vista de mes y la vista de lista.
    \item \textbf{Búsqueda en vista de lista:} Verifica que la funcionalidad de búsqueda de tareas esté operativa en la vista de lista.
    \item \textbf{Filtro por categoría:} Comprueba que el usuario pueda filtrar tareas por categoría en la vista de lista.
\end{itemize}

\section{Pruebas de Categorías}

\subsection{Creación de Categorías (e2e/categories/create.spec.ts)}
\begin{itemize}
    \item \textbf{Crear nueva categoría:} Verifica que un usuario pueda crear una nueva categoría proporcionando un nombre y seleccionando un color.
    \item \textbf{Crear categoría con color personalizado:} Asegura que se pueda crear una categoría utilizando un color personalizado.
\end{itemize}

\subsection{Eliminación de Categorías (e2e/categories/delete.spec.ts)}
\begin{itemize}
    \item \textbf{Eliminar categoría sin tareas:} Comprueba que una categoría que no tiene tareas asociadas pueda ser eliminada.
    \item \textbf{Confirmación antes de eliminar:} Verifica que se muestre un diálogo de confirmación antes de proceder con la eliminación de una categoría.
\end{itemize}

\subsection{Edición de Categorías (e2e/categories/edit.spec.ts)}
\begin{itemize}
    \item \textbf{Editar categoría existente:} Asegura que un usuario pueda editar el nombre y el color de una categoría existente.
    \item \textbf{Cambiar solo el color:} Verifica que se pueda cambiar únicamente el color de una categoría sin modificar su nombre.
\end{itemize}

\section{Pruebas de Tareas}

\subsection{Recomendaciones de IA (e2e/tasks/ai-recommendations.spec.ts)}
\begin{itemize}
    \item \textbf{Generar recomendaciones:} Comprueba que el sistema pueda generar recomendaciones de IA para una tarea específica.
\end{itemize}

\subsection{Creación de Tareas (e2e/tasks/create.spec.ts)}
\begin{itemize}
    \item \textbf{Crear nueva tarea:} Verifica que un usuario pueda crear una nueva tarea con un título, categoría y descripción.
    \item \textbf{Crear tarea con fecha de inicio:} Asegura que se pueda crear una tarea especificando una fecha de inicio.
\end{itemize}

\subsection{Edición de Tareas (e2e/tasks/edit.spec.ts)}
\begin{itemize}
    \item \textbf{Editar tarea existente:} Comprueba que un usuario pueda editar el título y la descripción de una tarea existente.
\end{itemize}

\section{Pruebas de Experiencia de Usuario}

