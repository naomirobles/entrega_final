                                                


%%%%%%%%%%%%%%%%% CUERPO %%%%%%%%%%%%%%%%%%%%%%%%
\section{Requerimientos Funcionales}
Los requisitos de \textit{software} establecen las especificaciones detalladas que definen qué debe hacer un sistema, cómo debe 
 comportarse y bajo qué condiciones debe operar \autocite{Requirements2024}. Este documento  tiene como propósito establecer de
 manera precisa y completa todas las funcionalidades, características y restricciones que deberá cumplir el sistema de gestión de
 tareas académicas y profesionales, garantizando así el desarrollo de una solución que satisfaga plenamente las necesidades identificadas. 
\begin{table}[H]
    \centering
    \begin{tablaRequisitos}{Requisitos Funcionales (Bloque 1)}
        \seccionTabla{1. Cuenta e Inicio de Sesión}
        \filaBlanca \textbf{Clave} & \textbf{Descripción} \\
        \filaGris RF1.1 & El sistema permite al usuario crear una cuenta ingresando correo electrónico, contraseña, país de origen y último grado de estudios. \\
        \filaBlanca RF1.2 & El sistema permite al usuario iniciar sesión con su correo electrónico y contraseña registradas. \\
        \filaGris RF1.3 & El sistema permite al usuario recuperar su contraseña mediante un correo electrónico de recuperación. \\
        \filaBlanca RF1.4 & El sistema permite al usuario ver y actualizar su perfil, incluyendo nombre, foto de perfil, país de origen y último grado de estudios. \\
        \filaGris RF1.5 & El sistema permite al usuario terminar su sesión. \\
        
        \seccionTabla{2. Registro de Tareas}
        \filaBlanca RF2.1 & Dado un usuario autenticado, al dar clic en "Nueva tarea", el sistema abre un modal donde el usuario puede ingresar el título, descripción y fecha de vencimiento. \\
        \filaGris RF2.2 & Tras ingresar los datos de la tarea, al dar clic en "Guardar", el sistema almacena la tarea con la etiqueta de "Todo". \\
        \filaBlanca RF2.3 & El usuario puede especificar el plazo (días de trabajo) además de la fecha límite. \\
        \filaGris RF2.4 & Vista mensual con etiquetas/contadores de tareas y acceso al detalle al hacer clic. \\
        \filaBlanca RF2.5 & El sistema permite al usuario modificar tareas. \\
        \filaGris RF2.6 & El sistema permite al usuario eliminar tareas.
    \end{tablaRequisitos}
    \label{tab:requisitos-funcionales-1}
\end{table}

\begin{table}[H]
    \centering
    \begin{tablaRequisitos}{Requisitos Funcionales (Bloque 2)}
        \seccionTabla{2. Registro de Tarea (Continuación)}
        \filaBlanca \textbf{Clave} & \textbf{Descripción} \\
        \filaGris RF2.7 & El usuario puede crear categorías personalizadas al registrar o editar tareas. \\
        \filaBlanca RF2.8 & El sistema permite visualizar tareas organizadas por categorías. \\
        \filaGris RF2.9 & Selección de categorías existentes para clasificación de tareas. \\
        \filaBlanca RF2.10 & Las tareas con plazo se muestran como etiquetas sobre el calendario en los días especificados. \\
        \filaGris RF2.11 & El sistema permite marcar tareas como completadas. \\
        \filaBlanca RF2.12 & Las tareas completadas muestran una línea horizontal sobre el título en la vista de calendario. \\

        \seccionTabla{3. Recomendación de recursos para consulta}
        \filaGris RF3.1 & Generación automática de recomendaciones (enlaces, videos, artículos) al agregar tareas. \\
        \filaBlanca RF3.2 & Recomendaciones basadas en el título de la tarea y el grado académico del usuario. \\
        \filaGris RF3.3 & El usuario puede guardar sugerencias para que aparezcan en el detalle de la tarea. \\
        \filaBlanca RF3.4 & Sistema de feedback para que el usuario indique si una recomendación fue útil.
    \end{tablaRequisitos}
    \caption{Requisitos funcionales consolidados para el sistema de gestión de tareas académicas y profesionales}
    \label{tab:requisitos-funcionales-2}
\end{table}

%--------------

\pagebreak

\section{Requerimientos No Funcionales}

\begin{table}[H]

    \centering

    \begin{tabular}{|p{3cm}|p{12cm}|}

    \hline

    \multicolumn{2}{|c|}{\textbf{Requerimientos No Funcionales (RNF)}} \\

    \hline

    \textbf{Categoría} & \textbf{Descripción} \\

    \hline

    Rendimiento & \textbf{RNF1} Las operaciones de interacción básica en la interfaz (abrir modal, guardar/editar tarea, navegar entre vistas) deberán responder en menos de \textbf{300 ms} en el 95\% de las peticiones bajo carga nominal. \\

    \hline

    Disponibilidad & \textbf{RNF2} El sistema deberá garantizar una disponibilidad mínima del \textbf{99.5\%} mensualmente (excluyendo ventanas de mantenimiento programado). \\

    \hline

    Seguridad & \textbf{RNF3} Autenticación segura (contraseñas almacenadas con hashing). \newline

                \textbf{RNF3} Todas las comunicaciones entre cliente y servidor deben estar cifradas mediante protocolo HTTPS. \newline

                \textbf{RNF3} Los usuarios deben autenticarse mediante credenciales seguras y el sistema debe implementar bloqueo tras 3 intentos fallidos. \\

    \hline

    Usabilidad & \textbf{RNF4} Interfaz intuitiva y consistente; realizar las principales acciones (crear tarea, marcar completada, editar) en \textbf{máximo 3 clics}. \newline

                \textbf{RNF4} Soporte móvil y escritorio. \\

    \hline

    Compatibilidad & \textbf{RNF5} Funcionamiento garantizado en navegadores modernos: Chrome, Firefox, Edge y Safari; versión adaptada para navegadores móviles equivalentes. \\

    \hline

    Mantenibilidad & \textbf{RNF6} Código modular y documentado.\\

    \hline

    Respaldo y Rendimiento & \textbf{RNF7} Copias de seguridad automáticas de la base de datos con retención mínima de \textbf{30 días}. \newline

                             \textbf{RNF7} El sistema debe soportar al menos 100 usuarios concurrentes sin degradación del rendimiento.\\

    \hline

    Recomendaciones & \textbf{RNF8} Las recomendaciones generadas al crear una tarea deberán presentarse en pantalla en menos de \textbf{5 segundos}.\\

    \hline

    Legal y Ética & \textbf{RNF9} Incluir términos y condiciones y política de privacidad. \newline

                   \textbf{RNF9} Las recomendaciones deben evitar contenidos que violen derechos de autor o políticas de contenido inapropiado. \\

    \hline

    \end{tabular}

    \caption{Requerimientos no funcionales principales}

    \label{tab:nfr}

\end{table}

%%%%%%%%%%%%%%%
\section{Historias de Usuario}

% 1. Registro de tareas
\begin{tablaHistoria}{Historia: Registro de tareas}
    \filaBlanca Descripción & Como profesional, necesito asignar fechas límite y estimar el tiempo para planificar mejor mi agenda. \\
    \filaGris Resultado & Tareas programadas con fechas límite y tiempo estimado. \\
    \filaBlanca Criterio & El usuario establece fecha límite y duración; el sistema muestra indicadores visuales de vencimiento.
\end{tablaHistoria}
\vspace{1em}

% 2. Registro de usuarios
\begin{tablaHistoria}{Historia: Registro de usuarios}
    \filaBlanca Descripción & Como profesional, necesito registrarme para acceder a mi lista desde distintos dispositivos. \\
    \filaGris Resultado & Registro de usuario que sirve como enlace al almacenamiento de las tareas. \\
    \filaBlanca Criterio & Registro con correo y contraseña. Incluye ingreso de grado académico y país para personalización.
\end{tablaHistoria}
\vspace{1em}

% 3. Modificación
\begin{tablaHistoria}{Historia: Modificación de tareas existentes}
    \filaBlanca Descripción & Como usuario, necesito editar tareas para actualizar información o cambiar fechas según mis necesidades. \\
    \filaGris Resultado & Tarea modificada con la información actualizada inmediatamente. \\
    \filaBlanca Criterio & El usuario puede editar cualquier campo y guardar los cambios exitosamente.
\end{tablaHistoria}
\vspace{1em}

% 4. Eliminación
\begin{tablaHistoria}{Historia: Eliminación de tareas}
    \filaBlanca Descripción & Como usuario, necesito eliminar tareas que ya no son relevantes para mantener mi lista organizada. \\
    \filaGris Resultado & Tarea eliminada permanentemente del sistema. \\
    \filaBlanca Criterio & El usuario elimina mediante confirmación; la tarea desaparece de todas las vistas.
\end{tablaHistoria}
\vspace{1em}

% 5. Vista Mensual
\begin{tablaHistoria}{Historia: Visualización mensual de tareas}
    \filaBlanca Descripción & Como usuario, necesito ver mis tareas en un calendario mensual para entender mi carga de trabajo. \\
    \filaGris Resultado & Vista de calendario mostrando todas las tareas del mes. \\
    \filaBlanca Criterio & Calendario interactivo donde al hacer clic se ve el detalle o acceso para editar.
\end{tablaHistoria}
\vspace{1em}

% 6. Categorías
\begin{tablaHistoria}{Historia: Creación de categorías personalizadas}
    \filaBlanca Descripción & Como usuario, necesito crear categorías (trabajo, personal, etc.) para clasificar mis tareas. \\
    \filaGris Resultado & Nuevas categorías creadas y disponibles para asignar. \\
    \filaBlanca Criterio & Categorías con nombre único y color distintivo disponibles al crear o editar tareas.
\end{tablaHistoria}
\vspace{1em}

% 7. Filtrado
\begin{tablaHistoria}{Historia: Visualización selectiva de tareas}
    \filaBlanca Descripción & Como usuario, necesito filtrar mi lista por categorías para enfocarme en un tipo de tarea particular. \\
    \filaGris Resultado & Vista de tareas filtrada según la categoría seleccionada. \\
    \filaBlanca Criterio & Opción de seleccionar múltiples categorías y limpiar filtros fácilmente.
\end{tablaHistoria}
\vspace{1em}

% 8. Notas
\begin{tablaHistoria}{Historia: Información detallada de tareas}
    \filaBlanca Descripción & Como usuario, necesito agregar notas o instrucciones detalladas a mis tareas. \\
    \filaGris Resultado & Tareas con información adicional almacenada correctamente. \\
    \filaBlanca Criterio & Campo de texto libre para notas, visible al editar o revisar la tarea.
\end{tablaHistoria}
\vspace{1em}

% 9. Recomendaciones
\begin{tablaHistoria}{Historia: Sugerencias automáticas de recursos}
    \filaBlanca Descripción & Como estudiante, necesito recibir recomendaciones (videos, artículos) relacionados con mis tareas. \\
    \filaGris Resultado & Lista de recursos recomendados generados automáticamente. \\
    \filaBlanca Criterio & El sistema muestra 3-5 recursos basados en el título y descripción de la tarea.
\end{tablaHistoria}
\vspace{1em}

% 10. Recomendaciones Perfil
\begin{tablaHistoria}{Historia: Recomendaciones personalizadas por perfil}
    \filaBlanca Descripción & Como usuario, necesito recomendaciones adaptadas a mi nivel de estudios y especialización. \\
    \filaGris Resultado & Recomendaciones precisas según el grado académico del usuario. \\
    \filaBlanca Criterio & El sistema utiliza los datos del perfil académico para filtrar la complejidad de los recursos.
\end{tablaHistoria}
\vspace{1em}

% 11. Guardado de favoritos
\begin{tablaHistoria}{Historia: Almacenamiento de recursos útiles}
    \filaBlanca Descripción & Como usuario, necesito guardar recomendaciones favoritas para acceder a ellas fácilmente. \\
    \filaGris Resultado & Recomendaciones seleccionadas guardadas en el detalle de la tarea. \\
    \filaBlanca Criterio & Ícono de favorito que ancla el recurso de forma permanente en la tarea.
\end{tablaHistoria}
\vspace{1em}

% 12. Feedback
\begin{tablaHistoria}{Historia: Feedback de recomendaciones}
    \filaBlanca Descripción & Como usuario, necesito indicar si una sugerencia fue útil para mejorar futuras recomendaciones. \\
    \filaGris Resultado & Feedback registrado para ajustar el algoritmo de IA. \\
    \filaBlanca Criterio & Botones simples de "útil" / "no útil" que personalizan la experiencia futura.
\end{tablaHistoria}
\newpage